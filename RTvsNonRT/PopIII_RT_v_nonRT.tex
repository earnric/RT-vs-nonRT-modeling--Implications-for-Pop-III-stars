\documentclass[twocolumn]{aastex63}

\usepackage{hyperref} % hyperref works too
\urlstyle{same}  % (sf also works, for something more subtle than tt)
\usepackage{amsmath}

\usepackage{mathrsfs}
%\usepackage{natbib}

\usepackage[]{units} % Provides \nicefrac

\usepackage{threeparttable}
\usepackage{multirow}

\usepackage{soul}

\renewcommand{\textfraction}{.1}
\renewcommand{\floatpagefraction}{.9}

\usepackage{ulem}

%\geometry{letterpaper}                   		% ... or a4paper or a5paper or ... 
%\usepackage[parfill]{parskip}    		% Activate to begin paragraphs with an empty line rather than an indent
\usepackage{graphicx}				% Use pdf, png, jpg, or eps§ with pdflatex; use eps in DVI mode
								    % TeX will automatically convert eps --> pdf in pdflatex		
\usepackage{dblfloatfix}

\setlength{\textfloatsep}{4pt plus 1.0pt minus 2.0pt}

\def\etal{{\textit{et al.} }}
\def\eg{{\textit {e.g. }}}

%%%%%
%\setcounter{topnumber}{2}
%\setcounter{bottomnumber}{2}
%\setcounter{totalnumber}{4}
%\renewcommand{\topfraction}{0.85}
%\renewcommand{\bottomfraction}{0.85}
%\renewcommand{\textfraction}{0.15}
%\renewcommand{\floatpagefraction}{0.7}
%%%%%

\begin{document}

\title{Paper title ... The coolest thing you've ever read}
\shorttitle{Short Title}
\author{Richard Sarmento}
\affiliation{United States Naval Academy,  121 Blake Road, Annapolis, MD, 21402, USA}
\affiliation{School of Earth and Space Exploration, 
Arizona State University, 
P.O. Box 871404, Tempe, AZ, 85287-1404}

\author{Evan Scannapieco}
\affiliation{School of Earth and Space Exploration, 
Arizona State University, 
P.O. Box 871404, Tempe, AZ, 85287-1404}

\shortauthors{Sarmento et al.}

%%%%%%%%%%%%%%
%%%%%%%%%%%%%%
\begin{abstract}
Recap the story... 
\end{abstract}

\keywords{cosmology: theory, early universe -- galaxies: high-redshift, evolution -- stars: formation, Population III -- luminosity function -- turbulence}

%\date{}	% Activate to display a given date or no date  

%%%%%%%%%%%%%%
%%%%%%%%%%%%%%
\section{Introduction}
intro stuff... Study the effects of radiation in early star and galaxy formation... How is star formation affected by the addition of RT? 

Differences: Where are stars formed? When are they formed? Metallicity...

The work is structured as follows.  In Section 2 we describe our methods, including a brief discussion of the implementation of our subgrid model for following the evolution of the pristine gas fraction, our approach to halo finding, and the spectral energy distribution (SED) models used to compute the luminosity of our stars. In Section 3 we show that ... . We compare nonRT to RT...  Next, we focus on an analysis of ... . Conclusions are discussed in Section 4. 
%cosmology 7-year WMAP ?CDM+SZ+LENS best fit (Komatsu et al. 2011)

%%%%%%%%%%%%%%
%%%%%%%%%%%%%%
\section{Methods}

%%%%%%%%%%%%%%
\subsection{RAMSES}
We use \textsc{Ramses-RT} \citep{Rosdahl2013} for this work, a cosmological adaptive mesh refinement (AMR) simulation with coupled radiation hydrodynamics (RHD). \textsc{Ramses-RT} is an extension of \textsc{Ramses} \citep{Teyssier2001} that models the interactions between dark matter, stellar populations, and baryons via gravity and hydrodynamics. \textsc{Ramses-RT} additionally models stellar radiation and radiative transfer as well as non-equilibrium radiative heating and cooling. To keep the radiative transfer computations manageable, \textsc{Ramses-RT} groups photon energies into a small number of user-defined bins. The simulation also employs a reduced speed of light that allows the time step-size to be reasonable as compared to non-RT codes. The simulation advects photons between cells using a first-order moment method with full local M1 closure for the Eddington tensor \citep{LEVERMORE1984149}.

Hydrodynamic flux between cells is computed using a Harten--Lax--van Leer contact (HLLC) Riemann solver \citep{Toro:1994we}. It is used to advect the typical cell-centered gas variables as well as the hydro scalars added by \cite{Sarmento2016} that track the turbulent velocity, the pristine gas mass fraction, and the metals generated by Population III (Pop III) supernova (SN). 

Self-gravity is solved using the multigrid method along with the conjugate gradient method for levels $\ge 12$.  Stars and DM are modeled with collisionless particles and are evolved using a particle-mesh solver with cloud-in-cell interpolation \citep{Guillet_2011}. We assume an ideal gas Equation of State with $\gamma$ = 5/3.

The following sections describes the set up for the two simulations, {\em RT} and {\em nonRT}, used to generate our results. We also review some of the modifications made to \textsc{Ramses-RT} used to track the pristine fraction of gas and the mass fraction of Pop III SN generated metals in each cell.


\subsection{Setup}
We use a cosmology based on \cite{Komatsu_2011}.
%: $\Omega_{\rm M} = 0.267$, $\Omega_{\Lambda} = 0.733$, $\Omega_{\rm b} = 0.0449$, $h = 0.71$, $\sigma_8 = 0.801$, and $n = 0.96,$ , where $\Omega_{\rm M}$, $\Omega_{\Lambda}$, and $\Omega_{\rm b}$ are the total matter, vacuum, and baryonic densities, respectively, in units of the critical density; $h$ is the Hubble constant in units of 100 km/s; $\sigma_8$ is the variance of linear fluctuations on the 8 $h^{-1}$ Mpc scale; and $n$ is the ``tilt" of the primordial power spectrum \citep{Larson_2011}. 
These and the other relevant simulation parameters are summarized in Table~\ref{tab:simParams}.

\begin{deluxetable}{r|l|r}
%\tabletypesize{\footnotesize}
\tablecolumns{3} 
\tablecaption{
	\label{tab:simParams}
	Simulation parameters. All parameters are common to both simulations, {\tt RTsim} \& {\tt nonRTsim}, up to the ``RT Only'' section. } 
\tablehead{\colhead{Parameter} & \colhead{Value} & \colhead{Description} } 
\startdata
\hline
\multicolumn{3}{c}{Cosmology}  \\
\hline
$\Omega_{\rm M}$ & 0.267 & Total matter density \\
$\Omega_{\Lambda}$ & 0.733 & Dark energy density\\
$\Omega_{\rm b}$ & 0.0449 & Baryon density \\
h & 0.71 & Hubble const [100 Mpc/s/kpc]\\
$\sigma_8$ & 0.801 & Power spectrum normalization \\
n & 0.96 & Power spectrum index \\
%\hline
\multicolumn{3}{c}{Setup}  \\
\hline
$\rm \ell_{min}$ & 9 & Base grid size - $2^9$ = 512 \\
$\rm \ell_{max}$ & 15 & Max refinement level \\
$\rm \Delta x_{max}$ & 5.86 h$^{-1}$ & Course/initial grid size [ckpc]\\
$\rm \Delta x_{ave}$ & 91.6 h$^{-1}$ & Best Average grid size [$cpc$]\\
$\rm M_{DM}$ & 17,500 & Dark matter particle mass [$\rm M_{\odot}$]\\
%\hline
\multicolumn{3}{c}{Star formation}  \\
\hline
$\epsilon_{\star}$ & 0.10 & Star forming efficiency \\
\multirow{2}{*}{$n_{\star}$} & \multirow{2}{*}{0.05} & SP formation density  \\
{\hspace{1em}} &{\hspace{1em}} & threshold [$\rm n_p/cc$] \\
\multirow{2}{*}{$\delta_{\star}$} & \multirow{2}{*}{200} & SP formation overdensity  \\
{\hspace{1em}} &{\hspace{1em}} & threshold [$\overline{\rho}$] \\
$m_{\star}$ & 2650 & SP mass resolution [$\rm M_{\odot}$]\\
%\hline
\multicolumn{3}{c}{Feedback}  \\
\hline
$\eta_{SNII}$&0.10& Pop II SN fraction at 10 Myr\\
$\eta_{SNIII}$&0.99& Pop III SN fraction at 10 Myr\\
\hline
\multicolumn{3}{c}{RT Only}  \\
\hline
$f_c$ & 0.01 & Reduced speed-of-light \\
$f_{esc,\star}$ & 0.5 & Stellar radiation multiplier \\
%\hline
\multicolumn{3}{c}{RT photon bins}  \\
\hline
$\rm UV_{H_2}$ & 11.20 - 13.60 & Lyman-Werner photons\\   %\\ [-1.5ex]
$\rm UV_{HI}$ & 13.60 - 24.59 & Hydrogen ionizing\\   %\\ [-1.5ex]
$\rm UV_{HeI}$ & 13.60 - 24.59 & HeI ionizing\\   %\\ [-1.5ex]
$\rm UV_{HeII}$ & 54.42 - $\infty$ &  HeII ionizing\\   %\\ [-1.5ex]
%\hline
\multicolumn{3}{c}{SB99 Metallicity bins}  \\
\hline
\multirow{5}{*} Z & 0.0000 - $10^{-5}$ & Pop III stars \\
{\hspace{1em}}  & $10^{-5}$  - 0.0004 & \multirow{3}{*} {Mass fraction of metals} \\
{\hspace{1em}} & 0.0004 - 0.0040 &  \\
{\hspace{1em}} & 0.0040 - 0.0080 &   \\
{\hspace{1em}} & 0.0080 - 0.0200 &   \\ 
\enddata 
\end{deluxetable} 


We evolve two 3 $h^{-1}$ comoving Mpc (cMpc) on-a-side simulations to $z=6$. This volume is approximately that of the local Milky Way (MW) Group. The simulation that models stellar photons and raditive transfer is henceforth {\tt RTsim}, while the non-RT simluation is named {tt nonRTsim}.
We set the initial refinement level to $\ell_{min} = 9$ corresponding to an coarse (initial) grid resolution $\rm \Delta x_{max} = 5.86\; h^{-1}$ comoving kpc (ckpc) -- a reasonable compromise that provides improved resolution of the intergalactic medium (IGM) without creating an excessive computational load. We adopt a quasi-Lagrangian approach to refinement such that cells are refined as they become approximately 8x over-dense. This strategy attempts to keep the amount of mass in each cell roughly constant as the simulation progresses. 

Allowing for up to 6 additional levels of refinement results in a best average resolution of 91.6 $\rm h^{-1}$ comoving pc (cpc). The initial grid scale and simulation size sets the dark matter (DM) particle mass. For this simulation $M_{\rm DM} = 1.75 \times 10^{4}\, M_{\odot}.$
%However, we stop the simulations at $\rm z=6$ where the best resolution was 18.4 pc physical. Our refinement strategy coupled with the nature of the initial density field results in  the maximum refinement level, $\ell = 13$, being reached at $\rm z \approx 20$. This occurs early-on in one of the rare over-density peaks. This means the best physical resolution at $\rm z\approx 20$ is approximately 3.3 times that at $\rm z=6$ or about 5.4 pc physical.


Initial conditions were identical for both simulations and were generated using Multi-Scale Initial Conditions (MUSIC) \citep{2013ascl.soft11011H}. The initial gas metallicity was $Z = 0,$ the initial $H_{\rm 2}$ fraction was $10^{-6}$ \citep{2005MNRAS.363..393R}, and we define $Z_{\rm crit} = 10^{-5} Z_\odot$, the boundary between Pop III and Population II (Pop II) star formation. The nonlinear length scale at the end of the simulation, $z= 6$, was 38 h$^{-1}$ ckpc , corresponding to a mass of $1.7\times 10^{7}$ h$^{-1}$ $M_\odot$. 

%
%Gas heating and cooling is modeled using \cite{Rosdahl2013}
%
%The UV background is derived from \cite{1996ApJ...461...20H}.
%
%
\subsection{Star Formation}
Star particles (SPs) are created in regions of gas according to a Schmidt law \cite{} with 
\begin{equation}\label{eqn:sf}
\dot{\rho_{\star}} =  \epsilon_{\star} \frac{\rho_{gas}}{t_{\rm ff}} \theta(\rho_{gas}- \rho_{\rm th})
\end{equation}
where $\rho_{\rm gas} > n_{\star}$ the star forming density threshold. Here we set $n_{\star}= 0.05\, m_{\rm p}/cc$. Additionally, the Heaviside step function, $\theta(\rho_{gas} - \rho_{\rm th})$, guarantees star formation occurs only when the gas density also exceeds a threshold value $\rho_{\rm th} = 200\, \overline{\rho}$. Here, $\overline{\rho}$ is the mean gas density of the simulation. We set the star forming efficiency to $\epsilon_{\star} = 0.10$, an empirically derived value that results in reasonable agreement with the observed cosmic star formation rate \citep{Finkelstein_2016, Madau_2014}. The gas free fall time is $t_{\rm ff} = \sqrt{3 \pi /(32 G \rho)}$. SPs represent an initial mass function (IMF) of stars. The SP mass is set by the star-forming density threshold and our resolution resulting in $m_{\star} = n_{\star} \Delta x^{3}_{ave} \approx 2.6 \times 10^{3}\, M_{\odot}$. The final mass of each SP is drawn from a Poisson process such that it is a multiple of $m_{\star}$.

\subsection{Feedback}
For normal Pop II stars ($Z > Z_{\rm crit}$) we assume a Salpeter IMF such that 10\% of each SP's mass represent stars more massive than 8 $M_\odot$ and go supernova (SN) in 10 Myr \citep{Raskin_2008, Somerville_2008}. For Pop III SPs ($Z \le Z_{\rm crit}$), we assume a log-normal and top-heavy IMF such that 99\% explode within 10 Myr \citep{1973MNRAS.161..133L, tumlinson2006chemical, raiter2010predicted}. 

The impact of these SNe is parameterized by the mass fraction of ejecta, $\eta_{\rm SN}$, and the kinetic energy per unit mass of the explosion, $E_{\rm SN}$. We take $\eta_{\rm SN}=0.10$ and $E_{\rm SN}=10^{51}$ ergs/10 $M_{\odot}$ for all stars formed throughout the simulation. The fraction of new metals in SN ejecta is 0.15 even though metal yields and energy from Pop III stars are likely to have been higher \citep{2003ApJ...589...35S,2005ApJ...624L...1S}. We may explore different yields and the subsequent effect on stellar enrichment in future work.  

The RT simulation uses {\it Starburst99} \citep{2011ascl.soft04003L} to model stellar populations with metallicities between $Z = 0.00040$ and $Z = 0.05$ (approx 2.5x $Z_\odot$) with ages between $\approx$0 Myr and a Gyr. Pop III stars are modeled with $Z = 0.00040$. We bin stellar photons into 4 groups to account for molecular hydrogen dissociating, hydrogen ionizing and 2 levels of helium ionization. Again, see Table~\ref{tab:simParams}. 

... Other parameters for the RT simulation needs $f_{esc,\star} = 0.5$. $f_c = 0.01$ is what we use for the reduced speed of light

We did not black holes (BH) in our simulation since BH feedback is not likely to be significant for our very early galaxies \citep{Somerville_2008, Scannapieco_2004}.  For the nonRTsim, we set the reionization redshift to match the reionization of the RT simulation: $z_{\rm reion} = 9$. Finally, all magnitudes are in the AB system \citep{1983ApJ...266..713O}.

%
%%%%%%%%%%%%%%%
%\subsection{The Pristine Fraction and the Corrected Metallicity}\label{sec:PF}
%In order to more accurately model the fraction of Pop III stars created throughout cosmic time, we track two new metallicity-related quantities.  The \textit{pristine gas mass fraction}, $P$, models the mass fraction of gas with $Z < Z_{\rm crit}$ in each simulation cell. The evolution of this scalar tracks the time history of metal mixing within the cell such that when $P=0$ the entire cell has been polluted above $Z_{\rm crit}$. The scalar $P_{\star}$ records, for all time, the value of $P$ in star particles at the time they are spawned and indicates the mass fraction of the SP with $Z_{\star} < Z_{\rm crit}$. 
%
%A simple equation can be used to describe the evolution of the pristine gas fraction in simulation cells:
%\begin{align}\label{eq:selfConv}
%\frac{d P}{d t} = - \frac{n}{\tau_{\rm con}}  P(1-P ^{1/n}). 
%\end{align}
%This equation traces the evolution of $P$ as a function of $n$ and a timescale $\tau_{\rm con}$, which, in turn, are functions of the turbulent Mach number, $M$, and the average metallicity of the cell relative to the critical metallicity, $\overline Z /Z_{\rm crit}$ \citep{2010ApJ...721.1765P, 2012JFM...700..459P, 2013ApJ...775..111P, 2017ApJ...834...23S}. Modeling the decay of the pristine gas fraction allows us to track the formation of Pop III stars as a mass fraction of all stars created, even in cells with an average metallicity above critical. 
%
%Each SP in the simulation is tagged with the average metallicity of the medium from which it was born, $\overline Z \rightarrow \overline Z_{\star}$. Furthermore, by knowing the average metallicity, $\overline Z$ (or $\overline Z_{\star}$ for SPs), and the pristine gas fraction, $P$ ($P_{\star}$), we can better model the metallicity of the polluted fraction of gas (or stars). More explicitly, since $\overline Z$ represents the average metallicity of a parcel of gas, and the polluted fraction, $f_{\rm pol} \equiv 1-P$, models the fraction of gas that is currently polluted with metals, we can use the value of $f_{\rm pol}$ to predict the enhanced, or corrected, metallicity,
%\begin{equation}\label{eq:zcorr}
%\begin{aligned}
%Z = \frac{\overline Z} {f_{\rm pol}},
%\end{aligned}
%\end{equation}
%of the polluted fraction of gas in each simulation cell. Similarly, $Z_{\star}$ captures the corrected metallicity of SPs. As expected, when $f_{\rm pol} = 1$ the corrected metallicity is the average metallicity.
%
%The metallicity of the polluted fraction as described by Eqn. (\ref{eq:zcorr}) is only precise when all of the metals are contained in the polluted fraction. This is true only in regions where the pristine gas is first polluted by Pop III SNe. However, it is possible for some of the metals to be distributed in the pristine gas fraction defined as $0 \le Z < Z_{\rm crit}$. As discussed in \cite{2017ApJ...834...23S}, this results in a small uncertainty in the resulting corrected metallicity of our SPs that we will ignore in this work. However, we can easily bound the correction to metallicity. While equation (\ref{eq:zcorr}) captures the upper bound, the lower bound on the correction is 
%\begin{equation}\label{eq:lowerlim}
%\begin{aligned}
%Z = \frac{\overline Z - Z_{\mathbb {P}} P }{f_{\rm pol}},
%\end{aligned}
%\end{equation}
%where $Z_{\mathbb {P}} = Z_{\rm crit} = 10^{-5} Z_{\odot}$ is the upper limit on the metallicity of the pristine gas. If the pristine fraction has $Z_{\mathbb {P}}=0$, as it would when polluting the primordial gas, we recover equation (\ref{eq:zcorr}).  Even when considering this uncertainty, the corrected metallicity, $Z$, allows us to more accurately model the metallicity of our gas and SPs than would be possible using the average metallicity alone. 
%
%Lastly, we note that we do not create polluted stars when $f_{\rm pol} < 10^{-5}$. In this case, we assume that all stars formed in the cell are Pop III since only a tiny fraction of the cell is polluted with metals. While this may seem arbitrary, it is used for convenience as such a small fraction of Pop II stars does not detectably contribute to the luminosity of our galaxies over the entire redshift range analyzed.
%
%\begin{figure}[h]
%\begin{center}
%\begin{tabular}{cc}
%\includegraphics[width=1\columnwidth]{./exampleZ}
%\end{tabular}
%\caption{An example (2D) distribution of metal concentrations across a set of AMR cells. While the average metallicity, $\overline{Z}$, across these four cells is greater than $Z_{\rm crit}$, there are clearly regions that are still pristine. We track a new scalar, $P$, in each AMR cell to quantify the fraction of gas with $Z < Z_{\rm crit}$. In conjunction, $P$ and $\overline{Z}$ allow us to better model the actual metallicity of the polluted fraction (as described in the text).}
%\label{fig:exz}
%\end{center}
%\end{figure}
%

%%%%%%%%%%%%%%
\subsection{Halo Finding}
Nothing yet
%So halo finding... Need to come up with a way to track numbers through z.
%
%We use the AdaptaHOP halo finder by \cite{2004MNRAS.352..376A} to find star-forming regions in the simulation volume at each redshift of interest. Only halos with at least 100 DM particles, corresponding to a DM halo mass of $1.4 \times 10^{7}\, M_{\odot}$, are considered by AdaptaHOP. Groups of 20 particles are used to compute the local density of a candidate halo and only objects with a density 80 times the average total matter density are stored.
%
%Several of the more massive objects found by AdaptaHOP consist of more than one observationally distinguishable galaxy. Hence, we postprocessed the halos as follows. For each AdaptaHOP halo, we compute a mass, in stars, within a 3 kpc comoving sphere centered on the halo's coordinates. This typically corresponds to the core of the most massive galaxy in the field. Next, we iteratively compute the mass in larger concentric spheres about this core. At each step, we increase the radius by $10^{-1}$ arcsec converted to a proper distance (in kpc) at the galaxy's redshift. By using a redshift-dependent step size based on the observational reference frame, we can roughly determine the boundaries of our galaxies, assuming, as is possible with the HST, that objects on the order of 0.1 arcsec apart are distinguishable. We continue increasing the radius until the fractional change in enclosed mass is less than one part in $10^{4}$. Specifically, when $\nicefrac{\Delta M_{\rm enc,i}}{M_{\rm enc,i}} < 10^{-4}$, we consider the current radius to be the radius of a single galaxy. Figure~\ref{fig:galaxs} depict the galaxies associated with an unreprocessed AdaptaHOP halo (left) and the resolved galaxies (right) that result from using this procedure. The approach ensures we do not overrepresent bright objects by considering multiple galaxies as one when computing their luminosities.


%%%%%%%%%%%%%%
%%%%%%%%%%%%%%
\section{Results}

The RTsim generates stars earlier than the nonRTsim as seen in Figure~\ref{fig:sfrd}. This is primarily due to improved non-equilibrium cooling (including $H_2$ cooling) that is not performed in the nonRTsim. However these first stars also produce a prodigious number of UV photons that heats the gas in the RTsim. This feedback acts to moderate star formation as compared to the nonRTsim which quickly overtakes it in terms of the overall star formation rate.

\begin{figure}[h]
\begin{center}
\includegraphics[width=1\columnwidth]{./SFRD-compare-RT-nonRT}
\caption{The SFRD for the nonRTsim and the RTsim.}
\label{fig:sfrd}
\end{center}
\end{figure}


Both simulations generate SN that remove gas from the central star forming region of the halos. The radiation pressure and heating in the RTsim does this more efficiently, and earlier, substantially reducing star formation in the core of the $z=15$ proto-galaxy depicted in Figure~\ref{fig:halos}. {\tt show pressure plot along with density plot} As can be seen in that figure, radiative feedback also creates pressure fronts that induce star formation away from the core, especially early on in the galaxy's evolution. We see this in Figure~\ref{fig:halos} where star formation is very centralized in the nonRTsim and much more dispersed -- and in fact along the pressure fronts -- in the RT simulation. The fact that the Pop III SFR in the RT simulation is higher than in the nonRTsim is due to this pressure-induced collapse that occurs in areas of gas that are 1) not thoroughly mixed yet and 2) still pristine (orange and green lines, respectively). Overall, there is a factor of 3 more stellar mass in the nonRTsim than in the RT simulation.

So more efficient cooling coupled with pressure-induced compression creates more Pop III stars early on.

Later -- situation reversed. Central regions and regions of unmixed gas are still dense and cool enough to continue to form Pop III while in the RT sim, the radiative heating and initially higher star formation have combined to puff up the gas, heating and smoothing it out. The amount of gas available for star formation is reduced -- as well as the density of that gas in/around the halo. So Pop III star formation continues in the outskirts of the nonRT halos, while it is effectively quenched in the RT sim.

\subsection{mass metallicity}
\subsection{luminosity func}
\subsection{CEMP-no}

%%%%%%%%%%%%%%
%%%%%%%%%%%%%%
\section{Conclusions}
We have used a large-scale cosmological simulation to study high-redshift galaxies and 

At high redshift, radiative feedback from stars reduces the overall/global SFR. However, radiation pressure seems 
%%%%%%%%%%%%%%
%%%%%%%%%%%%%%
\acknowledgments
We would like to thank ... We would also like to thank ... This work was supported by the USNA... The simulations and much of the analysis for this work was carried out on the USNA Advanced Research Cluster (ARC) and on the PSC Bridges2 Supercomputer at PSC. We would also like to thank the NASA High-End Computing Capability support team.

\software{\textsc{ramses} \cite{2010ascl.soft11007T}, MUSIC \citep{2013ascl.soft11011H}, pynbody \cite{2013ascl.soft05002P}, Starburst99 \cite{2011ascl.soft04003L}}

\clearpage
\pagebreak

%\appendix
\section{Appendix}
...
%
%Both simulations used the same parameters, including the star formation density threshold and were able to resolve star forming regions with 4 or more cells.

%\begin{figure}[h]
%\begin{center}
%\includegraphics[width=1\columnwidth]{./Compare}
%\caption{The SFRD for the fiducial run in \cite{2017ApJ...834...23S} and a run performed at half of that resolution. While there are inevitable differences between simulations due to the different resolutions, the subgrid model successfully recovers the Pop III rate shortly after the start of star formation at $z\approx18$. This demonstrates that modeling the subgrid fraction of pristine gas effectively improves the resolution of Pop III star formation for the simulation.}
%\label{fig:comp}
%\end{center}
%\end{figure}

\clearpage % Needed to ensure pictures are placed BEFORE the bibliography.
% Bibliography.
\bibliographystyle{apj}
\bibliography{PopIII_RT_v_nonRT.bib}
\end{document}  



